\subsection{EAP}
\paragraph{} Il framework che viene usato per la comunicazione tra le componenti del sistema è EAP - Extensible Authentication Protocol \cite{rfc2}. I messaggi EAP tra Supplicant e Authenticator sono incapsulati nei frame IEEE 802.1x secondo il protocollo EAPOL - EAP Over Lan. La comunicazione tra Authenticator e Authentication Server veicola invece i messaggi EAP attraverso dei protocolli al di sopra del Data Link Layer, come RADIUS basato su UDP, descritto nel capitolo \ref{subsec:radius}.
\paragraph{} Quando un nuovo client si connette alla rete inizia questo scambio di informazioni:
\begin{enumerate}
	\item Authenticator $\Longrightarrow$ Supplicant: l'Authenticator trasmete ad intervalli regolari {\tt EAP Request Identity}, 
	\item Supplicant $\Longrightarrow$ Authenticator $\Longrightarrow$ Authentication Server: nel momento in cui il Supplicant riceve la richiesta risponde con un {\tt EAP Responde Identity} in cui manda la sua identità. Questo messaggio viene incapsulato in un pacchetto {\tt RADIUS access request} e inoltrato all'Authentication Server dall'Authenticator.
	\item Authentication Server $\Longrightarrow$ Authenticator $\Longrightarrow$ Supplicant: l'Authentication Server risponde all'Authenticator con un {\tt RADIUS access challenge}, l'Authenticator incapsula l'EAP in un EAPOL e lo trasmette al Supplicant. Lo scambio verrà ripetuto finchè necessario.
	\item Lo scambio di messaggi continua fino al momento in cui l'Authentication Server:  
	\begin{itemize}
		\item determina che non è possibile autenticare il Supplicant mandando il pacchetto {\tt EAP failure}, oppure
		\item stabilisce il successo del progesso di autenticazione mandando il pacchetto {\tt EAP success}.
	\end{itemize}
\end{enumerate}


