\section{SICUREZZA}
\label{sec:sicurezza}
\paragraph{}
In questo capitolo è analizzato il problema del controllo degli accessi, dell'autenticazione e della sicurezza dei dati nella rete. Gli accessi ad una rete aziendale devono essere controllati e rintracciabili per i dipendenti e per gli ospiti (personale esterno, consulenti, dottori) mentre deve essere impossibile accedervi da parti terze.
\paragraph{Principi di sicurezza delle reti}
La realizzazione di un sistema di sicurezza è strettamente legato ai meccanismi che regolano l'autenticazione delle parti in comunicazione e che garantiscono la confidenzialità dei dati in transito. Affinchè possa essere sicura la rete deve garantire:
\begin{itemize}
\item Confidenzilità: solo il mittente e il destinatario dovrebbero "capire" il contento del messaggio. Per questo scopo si utilizza la crittografia, il mittente cripta il messaggio, il destinatario lo decripta.
\item Autenticazione: processo con cui il mittente e il destinatario confermano la loro identità, garantendo che i messaggi ricevuti non siano modificati da una terza parte.
\item Integrità del messaggio: il mittente e il destinatario vogliono assicurare che il messaggio non si sia alterato nel transito sul canale fisico o successivamente.
\item Accesso e disponibilità: i servizi offerti dalla rete devono essere sempre accessibili e disponibili agli utenti.
\end{itemize} 
\subsection{Autenticazione tramite IEEE 802.1x}
\paragraph{} Un utente che vouole accedere ad una rete deve possedere delle credenziali di accesso (per esempio un account o un certifcato digitale) in modo che possa essere verifcata la sua identità; deve anche però poter stabilire in modo sicuro che il server o l'access point a cui si sta rivolgendo appartenga effettivamente ad una rete legittima, senza correre il rischio di fornire le proprie informazioni ad un utente non autorizzato. 
Nella realizzazione della rete aziendale del Nuovo Blocco Operatorio verrà utilizzato il protocollo di autenticazione 802.1x, andando ad aggiornare con questo sistema più sicuro l'intera rete preesistente.
\paragraph{IEEE 802.1x}
È stato scelto l'utilizzo di questo protocollo poichè si basa sul controllo delle porte di accesso alla rete LAN e fornisce un meccanismo di autenticazione ai dispositivi che vogliono collegarsi tramite uno switch o un access point alla rete locale, stabilendo un collegamento point-to-point e impedendo collegamenti non autorizzati alla rete. Si può trovare una descrizione più dettagliata qui \cite{csc1} e nell'RFC 3580 \cite{rfc1}.
Come si vede in figura \ref{fig:amb} l'IEEE 802.1x suddivide i componenti della rete in:
\begin{itemize}
    \item Supplicant: è il nuovo utente che sta cercando di connettersi alla rete.
    \item Authenticator o Network Access Server (NAS): è lo switch o l'access point con il quale il supplicant comunica.
    \item Authentication Server: è il server RADIUS - Remote Authentication Dial-in User Service che gestisce l'autenticazione.
\end{itemize}

\begin{figure}[h]
  \centering
  \includegraphics[width=0.5\textwidth]{ambiente8021x.png}
  \caption{Ambiente di esecuzione del protocollo IEEE 802.1x}
  \label{fig:amb}
\end{figure}

\subsection{EAP}
\paragraph{} Il framework che viene usato per la comunicazione tra le componenti del sistema è EAP - Extensible Authentication Protocol \cite{rfc2}. I messaggi EAP tra Supplicant e Authenticator sono incapsulati nei frame IEEE 802.1x secondo il protocollo EAPOL - EAP Over Lan. La comunicazione tra Authenticator e Authentication Server veicola invece i messaggi EAP attraverso dei protocolli al di sopra del Data Link Layer, come RADIUS basato su UDP, descritto nel capitolo \ref{subsec:radius}.
\paragraph{} Quando un nuovo client si connette alla rete inizia questo scambio di informazioni:
\begin{enumerate}
	\item Authenticator $\Longrightarrow$ Supplicant: l'Authenticator trasmete ad intervalli regolari {\tt EAP Request Identity}, 
	\item Supplicant $\Longrightarrow$ Authenticator $\Longrightarrow$ Authentication Server: nel momento in cui il Supplicant riceve la richiesta risponde con un {\tt EAP Responde Identity} in cui manda la sua identità. Questo messaggio viene incapsulato in un pacchetto {\tt RADIUS access request} e inoltrato all'Authentication Server dall'Authenticator.
	\item Authentication Server $\Longrightarrow$ Authenticator $\Longrightarrow$ Supplicant: l'Authentication Server risponde all'Authenticator con un {\tt RADIUS access challenge}, l'Authenticator incapsula l'EAP in un EAPOL e lo trasmette al Supplicant. Lo scambio verrà ripetuto finchè necessario.
	\item Lo scambio di messaggi continua fino al momento in cui l'Authentication Server:  
	\begin{itemize}
		\item determina che non è possibile autenticare il Supplicant mandando il pacchetto {\tt EAP failure}, oppure
		\item stabilisce il successo del progesso di autenticazione mandando il pacchetto {\tt EAP success}.
	\end{itemize}
\end{enumerate}



\subsection{Server RADIUS e AAA}
\label{subsec:radius}
\paragraph{} Il protocollo RADIUS - Remote Authentication Dial-In User Service è un protocollo basato su UDP che gestisce l'autenticazione delle connessioni remote, verificando sul database SQL la presenza delle credenziali, del certificato o del MAC address. Il servizio RADIUS è implementato dal software FreeRADIUS su una macchina virtuale tramite VMware ed eseguito sul cluster composto da 10 server fisici detti \textsc{host}, sui quali viene distribuita l'esecuzione delle 143 macchine virtuali dette \textsc{guest}. Questo meccanismo è illustrato nel capitolo \ref{sec:virtualizzazione}. Le risorse sono assegnate dinamicamente sulla base di un monitoring del carico delle macchine virtuali e dell’utilizzazione delle risorse dell'\textsc{host}.


\paragraph{AAA} Il database delle credenziali di accesso è contenuto in maniera centralizzata nel SAN (Storage Area Network) accessibile al server RADIUS da qualsiasi \textsc{host} su cui è in esecuzione.
L'autenticazione tramite RADIUS si basa sui principi di AAA (Autenticazione Autorizzazione e Accounting):
\begin{itemize}
    \item Autenticazione: fase in cui viene verificata la presenza e la correttezza delle credenziali di accesso. 
    \item Autorizzazione: regolazione dei diritti e delle modalità di accesso (es. autorizzazione per un tempo limitato)
    \item Accounting: mantenimento dei resoconti sull'attività dei soggetti autenticati che hanno avuto accesso al sistema.
\end{itemize}


\subsection{Metodo di autenticazione}
\paragraph{} Il protocollo IEEE 802.1x prevede più modalità di autenticazione alla rete. La rete ospedaliera non ha un accesso wireless proprio, ma ha una rete cablata che arriva fino alle torrette e alle prese di rete 503, presenti in ogni stanza. Ci sono quindi dei dispositivi come PC e stampanti sempre connessi alla rete, ma devono essere previste nuove connessioni per il personale che si trovi momentaneamente a lavorare all'interno dell'azienda ospedaliera. Analizziamo in seguito le due principali possibilità per un'autenticazione sicura e veloce.

\subsubsection{Autenticazione tramite credenziali}
\paragraph{} Questa opzione prevede che ogni dipendente dell'azienda sanitaria che deve avere accesso alla rete internet aziendale abbia delle credenziali composte da un Username (es. nome.cognome) e una password da lui modificabile. Adottando questo sistema si prevede uno scambio di messaggi EAP ed EAPoL tra il supplicant, il NAS (Network Attached Storage) e il server RADIUS. Se le credenziali inviate dal supplicant sono presenti nel database del server, l'utente sarà autenticato ed autorizzato, altrimenti lo switch imposta lo stato della porta sulla quale comunica su "non autorizzata".
\paragraph{Vantaggi} Il servizio così offerto autentica la persona e permette un accesso unificato su ogni dispositivo (PC della postazione di lavoro, smartphone, notebook). Vengono registrate le sessioni di lavoro, è possibile gestire autorizzazioni per dei gruppi di utenti ed è possibile assegnare delle credenziali temporanee per gli ospiti o nel caso di eventi organizzati. 
\paragraph{Svantaggi} Per la struttura ospedaliera è necessario collegare in rete solamente i PC delle postazioni di lavoro, non deve esere previsto collegare smartphone o notebook privati.

\subsubsection{Autenticazione tramite MAC address}
\paragraph{} Un estensione interessante del protocollo 802.1x è quella di poter autenticare un end device tramite l'indirizzo MAC univoco che possiede. Nel server RADIUS verrà quindi salvato l'elenco degli intirizzi noti della rete. All'atto del supplicant di autenticarsi, l'operazione andrà a buon fine solo se il suo indirizzo MAC è già presente nel database. E' importante notare come schematizzato in Figura \ref{fig:mab} che il servizio di MAB - MAC Address Bypass entra in funzione solamente dopo il fallimento o il timeout delle richieste EAPoL: questo determina un tempo di latenza elevato, risolvibile però impostando al minimo (2 secondi) il tempo di attesa della prima parte. Entra così in funzione MAB, il quale interroga tramite una query SQL il FreeRADIUS e determinerà l'autenticazione o la disabilitazione della porta. 

\begin{figure}[h]
  \includegraphics[width=\linewidth]{mabprocess.jpg}
  \caption{MAB come meccanismo di failover.}
  \label{fig:mab}
\end{figure}

\paragraph{Vantaggi} Eliminato il tempo di latenza per il retries dell'IEEE 802.1x si passa velocemente ad un accesso immediato ed automatico, che non prevede interazione con l'utente; l'usufruitore del servizio sarà abilitato ad accedere alla rete automaticamente. 
\paragraph{Svantaggi} Il servizio così offerto autentica il dispositivo e non l'utente che lo utilizza. Inoltre, poichè questa sarà un'integrazione della rete preesistente, sono stati recuperati i MAC address tramite uno script eseguito sul router gateway. L'elenco dei MAC address presente è stato inserito nel database, ci è però impossibile sapere se sono tutti validi o se uno di questi appartiene ad una terza parte malintenzionata. Prendiamo quindi tutti i MAC address salvati nel router come buoni.






\subsection{Implementazione finale}
\paragraph{} Date le condizioni sopra descritte il metodo di autenticazione della rete scelto è il MAB: è il più adatto alla realizzazione della rete richiesta, il più pratico e "invisibile" all'utente. 
La configurazione da dare agli switch è descritta nell'appendice \ref{appendix:swconfig}. La configurazione da noi creata è stata fatta su uno Switch Cisco Catalyst 3750, Version 12.2(25)SEE3.

