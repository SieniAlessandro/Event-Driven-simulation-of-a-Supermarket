\subsection{Autenticazione tramite IEEE 802.1x}
\paragraph{} Un utente che vouole accedere ad una rete deve possedere delle credenziali di accesso (per esempio un account o un certifcato digitale) in modo che possa essere verifcata la sua identità; deve anche però poter stabilire in modo sicuro che il server o l'access point a cui si sta rivolgendo appartenga effettivamente ad una rete legittima, senza correre il rischio di fornire le proprie informazioni ad un utente non autorizzato. 
Nella realizzazione della rete aziendale del Nuovo Blocco Operatorio verrà utilizzato il protocollo di autenticazione 802.1x, andando ad aggiornare con questo sistema più sicuro l'intera rete preesistente.
\paragraph{IEEE 802.1x}
È stato scelto l'utilizzo di questo protocollo poichè si basa sul controllo delle porte di accesso alla rete LAN e fornisce un meccanismo di autenticazione ai dispositivi che vogliono collegarsi tramite uno switch o un access point alla rete locale, stabilendo un collegamento point-to-point e impedendo collegamenti non autorizzati alla rete. Si può trovare una descrizione più dettagliata qui \cite{csc1} e nell'RFC 3580 \cite{rfc1}.
Come si vede in figura \ref{fig:amb} l'IEEE 802.1x suddivide i componenti della rete in:
\begin{itemize}
    \item Supplicant: è il nuovo utente che sta cercando di connettersi alla rete.
    \item Authenticator o Network Access Server (NAS): è lo switch o l'access point con il quale il supplicant comunica.
    \item Authentication Server: è il server RADIUS - Remote Authentication Dial-in User Service che gestisce l'autenticazione.
\end{itemize}

\begin{figure}[h]
  \centering
  \includegraphics[width=0.5\textwidth]{ambiente8021x.png}
  \caption{Ambiente di esecuzione del protocollo IEEE 802.1x}
  \label{fig:amb}
\end{figure}
