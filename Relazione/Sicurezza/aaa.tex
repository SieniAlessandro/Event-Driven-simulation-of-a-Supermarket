\subsection{Server RADIUS e AAA}
\label{subsec:radius}
\paragraph{} Il protocollo RADIUS - Remote Authentication Dial-In User Service è un protocollo basato su UDP che gestisce l'autenticazione delle connessioni remote, verificando sul database SQL la presenza delle credenziali, del certificato o del MAC address. Il servizio RADIUS è implementato dal software FreeRADIUS su una macchina virtuale tramite VMware ed eseguito sul cluster composto da 10 server fisici detti \textsc{host}, sui quali viene distribuita l'esecuzione delle 143 macchine virtuali dette \textsc{guest}. Questo meccanismo è illustrato nel capitolo \ref{sec:virtualizzazione}. Le risorse sono assegnate dinamicamente sulla base di un monitoring del carico delle macchine virtuali e dell’utilizzazione delle risorse dell'\textsc{host}.


\paragraph{AAA} Il database delle credenziali di accesso è contenuto in maniera centralizzata nel SAN (Storage Area Network) accessibile al server RADIUS da qualsiasi \textsc{host} su cui è in esecuzione.
L'autenticazione tramite RADIUS si basa sui principi di AAA (Autenticazione Autorizzazione e Accounting):
\begin{itemize}
    \item Autenticazione: fase in cui viene verificata la presenza e la correttezza delle credenziali di accesso. 
    \item Autorizzazione: regolazione dei diritti e delle modalità di accesso (es. autorizzazione per un tempo limitato)
    \item Accounting: mantenimento dei resoconti sull'attività dei soggetti autenticati che hanno avuto accesso al sistema.
\end{itemize}

