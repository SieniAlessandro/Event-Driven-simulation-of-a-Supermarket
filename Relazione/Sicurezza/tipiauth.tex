\subsection{Metodo di autenticazione}
\paragraph{} Il protocollo IEEE 802.1x prevede più modalità di autenticazione alla rete. La rete ospedaliera non ha un accesso wireless proprio, ma ha una rete cablata che arriva fino alle torrette e alle prese di rete 503, presenti in ogni stanza. Ci sono quindi dei dispositivi come PC e stampanti sempre connessi alla rete, ma devono essere previste nuove connessioni per il personale che si trovi momentaneamente a lavorare all'interno dell'azienda ospedaliera. Analizziamo in seguito le due principali possibilità per un'autenticazione sicura e veloce.

\subsubsection{Autenticazione tramite credenziali}
\paragraph{} Questa opzione prevede che ogni dipendente dell'azienda sanitaria che deve avere accesso alla rete internet aziendale abbia delle credenziali composte da un Username (es. nome.cognome) e una password da lui modificabile. Adottando questo sistema si prevede uno scambio di messaggi EAP ed EAPoL tra il supplicant, il NAS (Network Attached Storage) e il server RADIUS. Se le credenziali inviate dal supplicant sono presenti nel database del server, l'utente sarà autenticato ed autorizzato, altrimenti lo switch imposta lo stato della porta sulla quale comunica su "non autorizzata".
\paragraph{Vantaggi} Il servizio così offerto autentica la persona e permette un accesso unificato su ogni dispositivo (PC della postazione di lavoro, smartphone, notebook). Vengono registrate le sessioni di lavoro, è possibile gestire autorizzazioni per dei gruppi di utenti ed è possibile assegnare delle credenziali temporanee per gli ospiti o nel caso di eventi organizzati. 
\paragraph{Svantaggi} Per la struttura ospedaliera è necessario collegare in rete solamente i PC delle postazioni di lavoro, non deve esere previsto collegare smartphone o notebook privati.

\subsubsection{Autenticazione tramite MAC address}
\paragraph{} Un estensione interessante del protocollo 802.1x è quella di poter autenticare un end device tramite l'indirizzo MAC univoco che possiede. Nel server RADIUS verrà quindi salvato l'elenco degli intirizzi noti della rete. All'atto del supplicant di autenticarsi, l'operazione andrà a buon fine solo se il suo indirizzo MAC è già presente nel database. E' importante notare come schematizzato in Figura \ref{fig:mab} che il servizio di MAB - MAC Address Bypass entra in funzione solamente dopo il fallimento o il timeout delle richieste EAPoL: questo determina un tempo di latenza elevato, risolvibile però impostando al minimo (2 secondi) il tempo di attesa della prima parte. Entra così in funzione MAB, il quale interroga tramite una query SQL il FreeRADIUS e determinerà l'autenticazione o la disabilitazione della porta. 

\begin{figure}[h]
  \includegraphics[width=\linewidth]{mabprocess.jpg}
  \caption{MAB come meccanismo di failover.}
  \label{fig:mab}
\end{figure}

\paragraph{Vantaggi} Eliminato il tempo di latenza per il retries dell'IEEE 802.1x si passa velocemente ad un accesso immediato ed automatico, che non prevede interazione con l'utente; l'usufruitore del servizio sarà abilitato ad accedere alla rete automaticamente. 
\paragraph{Svantaggi} Il servizio così offerto autentica il dispositivo e non l'utente che lo utilizza. Inoltre, poichè questa sarà un'integrazione della rete preesistente, sono stati recuperati i MAC address tramite uno script eseguito sul router gateway. L'elenco dei MAC address presente è stato inserito nel database, ci è però impossibile sapere se sono tutti validi o se uno di questi appartiene ad una terza parte malintenzionata. Prendiamo quindi tutti i MAC address salvati nel router come buoni.





