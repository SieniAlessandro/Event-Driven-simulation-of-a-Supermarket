\subsection{Modeling}
\paragraph{} 
In this project we had to model two different types of systems. First, we have a fixed number of servers (during a simulation), which represents tills in a real supermarket, and a single queue, where the customers arrive and wait to be served. In this scenario the servers have the same priority and they start to serve a customer when they finished the previous one and the queue is not empty (we simulated a work conserving system). The second scenario is different, because each server has its own queue and it serves only customers in line in that queue. In this scenario, considering real life, there is a priority to reach a specific queue, because in real life the customers, in the most of cases, want to walk as less as possible, and to model that, in case of a equal lenght for more than one queue, the customer will reach the server that is more near to him. With this behavior the nearest tills are busier than the most distant tills. 