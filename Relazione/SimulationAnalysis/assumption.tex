\subsection{A priori assumption}
\paragraph{} 
The assumption made to have a good model to study the behavior of the system is the following:
\begin{enumerate}
	\item The system has ideal  delay = 0 s among all the network nodes and in case a) a fixed delay equal to $\Delta \cdot j$
	\item Queue hasn't a maximum size, all customers that enter the system will be served
	\item The system at simTime zero is empty
	\item Distributions of inter-arrival time (IAT) and service time (ST) are independent from the workload
	\item The pseudo-random numbers generated in the simulation by Omnet can be considered IID
\end{enumerate}
In order to model the system is  under the hypothesis of continuous-time Markov Chains \[M/M/C/\infty/\infty\] where:
\begin{enumerate}
	\item C is finite because the number of tills can variate, but stay constant in a single simulation
	\item The queue is infinite because the tills can serve all the people.
	\item The population is infinite because before the simulation we don’t know how many people will arrive at each till.
\end{enumerate}
