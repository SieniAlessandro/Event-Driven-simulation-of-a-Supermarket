\section{Scenario Analysis}
\paragraph{}
In this project, for each suitable configuration we have selected different seeds, one for each run. This action has been implemented on inter-arrival time of customers and service time, having a different seed for each run is necessary for the idea of IIDness. Seeds represent inter-arrival rate for the queue and service rate for the till. Each queue and each till have a different seed. Moreover, all statistical data are extracted from the simulator when it is in the steady state, where the parameters of interest has shown relatively constant behaviour. In general, high utilization and lowest delay and queuing time is always the case of interest, since the arrival rate and service rate, $\lambda$ and $\mu$, can lead to the overbalance of people in the system, that would cause too long waiting times. These two parameter has been the point of debate which helped us finding more suitable configurations and more accurate system limits. It is possible to have a same utilization by having different values of parameters e.g. increasing arrival time and decreasing service time, but this is not suitable for the logic of the supermarket.    

\subsection{Exponential Scenario}
\paragraph{} The first scenario analyzed is based on Exponential RVs, which has this PDF:
\begin{centering}
\begin{equation}
f(x) = \begin{cases}
 \lambda e^{-\lambda x} &  x\geq 0 \\ 
 0 & x<0 
\end{cases}
\end{equation}
\end{centering}

\paragraph{} and this CDF:

\begin{centering}
\begin{equation}
F(x) = \begin{cases}
 1- \lambda e^{-\lambda x} &  x\geq 0 \\ 
 0 & x<0 
\end{cases}
\end{equation}
\end{centering}
\paragraph{} QUESTA ROBA SECONDO ME ERA STATA SCRITTA NEL PARAGRAFO PRECEDENTE Since in this simulation we want to study how the system works in different scenarios, we have used one seed for the arrival time of the customer and one seed for each till (a seed for each server). This choice guarantees us that the random variable present in our system are IID, because they are generated using different seed.
The only fixed number present in our simulation is the time necessary for a customer to reach a till, because in a considerable amount of time, the position of the till in a supermarket are not changing.


\subsubsection{Parameter analysis}
\paragraph{} 
In this scenario, the inter-arrival time is represent by $\lambda$ [s]; that stand for how many seconds elapse on average between one customer's arrival and another. To find out how many people arrive per second, it's necessary to compute the arrival's frequency $\frac{1}{\lambda} [\frac{1}{s}]$ (inter-arrival rate). The same is true for $\mu$ [s] (service time) that stand for how many seconds elapse, on average, between one customer's service and another and for $\frac{1}{\mu} [\frac{1}{s}]$ (service rate), the frequency of service as how many people are served per second. It's realistic think that a new customer arrives each, at least every 20 seconds, at most every 300 seconds and that a customer is served at least every 10 seconds, atmost every 240 seconds. We obtain $\lambda \in [20, 300]$ and $\mu \in [10, 240]$. \\
\paragraph{} Moreover we need to respect the stability condition $\lambda < C\cdot \mu$ where C is the number of tills. This could be expected, since \lambda is the arrival and \mu is the service at high loads. In fact, the utilization is $\rho = \frac{\lambda }{C\cdot \mu}$ and the stability condition is $\rho < 1$  


Puoi parlare del tempo di ritardo, indicandolo come valore fisso, che è direttamente proporzianale al numero di cassa, che nel nostro scenario indica anche la distanza della cassa dal "punto di partenza"
 Oppure indicare come parametro il numero di casse, mostrando eventualmente, magari in base ad esperienze reali, quali sono i numeri di casse (C) maggiornamente rilevanti, ovvero quelli che hanno maggior probabilità di essere presneti in un supermercato
 Per esempio : Noi abbiamo deciso di analizzare un sistema il cui numero di casse varia da 2 a 30, ben consci però che numeri molto elevati (es sopra il 10) sono difficilmente riscontrabili, e per tanto le nostre analisi si sono focalizzate maggiormente su questo sotto-intervallo
 Stesso discorso lo puoi fare per lambda e mu
 In generale secondo me è molto importante indicare i range selezionati ed i motivi che ci hanno portato fare quelle scelte, ed in particolare le motivazioni dietro ad analisi specifiche
%During the process of fixing an adequate value for interarrival time and service time, an interesting behavior has been observed: while health regeneration
%seemed not to beaffecting too much the utilization of the system, it caused the
%system to get unstable when the utilization started growing above80% ; there-
%fore we decided to perform a 2 k r factorial analysis to verify the actual impact
%of the parameters on utilization:
%as we can see, the highest effect is given by service rate(insert percentage), fol-
%lowed by minionIAT(percentage), bossIAT(percentage) and least from health
%regen(percentage).This is what we expected, since the definition of utilization
%in case of two different job queues (ignoring, for sake of simplicity, health regen-
%eration)is:
%ρ = (λ 1 + λ 2 )/µ
%and this means that the contribution in the nominator is somehow divided
%between the two generators.Concerning the instability caused by health regen-
%eration, it is explainable by the fact that decreasing the mean value for the boss
%intersend time increases the probability that a boss arrives “too early”, causing
%preemption in a busysystem that has to handle the new boss and the previous,
%still not handled and “reinforced”by health regeneration workload.
\input{./ScenarioAnalysis/lognormal.tex}
\input{./ScenarioAnalysis/grafici.tex}
%\subsection{Cablaggio strutturato}

\paragraph{Livello fisico}
Il livello fisico ha il compito di trasmettere un flusso di dati, bits, attraverso un collegamento fisico. Deve definire le modalità di invio dei singoli bits. 
Questo è l'unico livello che riguarda direttamente l'hardware, infatti è qui che vengono definite le caratteristiche dei cavi ed il loro persorso (cablaggio). Viene inoltre decisa la tipologia di trasmissione dei segnali (rame, fibra, etc) e l'eventuale trasmissione simultanea (half duplex ovvero full duplex).

\paragraph{Cavi}
Il cablaggio strutturato viene fatto nelle aziende dove si ha la necessità di collegare i PC a server e periferiche e di fornire collegamenti per ogni tipo di dispositivo elettronico, nel nostro caso dispositivi medicali e apparecchiature mediche.
Per il cablaggio strutturato su più piani consideriamo le categorie di cavi utilizzabili:
\begin{itemize}
    \item Cavi di categoria 5e (5 enhanced): supportano trasmissioni fino a 1 Gibabit per distanze minori di 100m e frequenza massima 250MHz.
    \item Cavi di categoria 6: supportano trasmissioni costanti alla velocità di 5 Gigabit per distanze minori di 100m e frequenza massima 500 MHz.
    \item Cavi di categoria 7: cavi con doppia schermatura, supportano trasmissioni alla velocità di 10 Gigabit Ethernet per distanze minori di 100m e frequenza massima 600 MHz. 
    \item Cavi di categoria 7A: per frequenze fino a 1000MHz, supporteranno in futuro una ethernet a 40Gigabit 40Gbase-T entro 50m e una 100Gigabit ethernet a 15 metri. 
\end{itemize}

Il cablaggio può essere fatto in rame con cavi di tipo:
\begin{itemize}
  \item Cavo UTP (Unshielded Twisted Pair, "doppino non schermato"): flessibile e senza schermatura, è composto dalla guaina di plastica protettiva, i quattro doppini e i connettori RJ45.
  \item Cavo FTP (Foiled Twisted Pair, "doppino avvolto"): con schermatura singola, racchiude i quattro doppini e scherma parzialmente da interferenze elettromagnetiche. 
  \item Cavo STP (Shielded Twisted Pair, "doppino schermato"): con doppia schermatura, la guaina schermante è presente sia attorno ad ogni singolo doppino sia attorno alle quattro coppie, come nel caso precedente. Scherma da ogni interferenza ma a discapito della flessibilità del cavo.
\end{itemize}

\paragraph {Cablaggio verticale}
Il cablaggio verticale collega gli armadi MDF (Main Distribution Frame) a:
\begin{itemize}
  \item agli armadi IDF (Intermediate Distribution Frame) e quindi gli switch di core agli switch di distribution e access. 
  \item al core della rete situato nell'edificio principale.
\end{itemize}
Il cablaggio verticale è realizzato in fibra con quattro cavi multimodali WS-G5484 dal core switch dell'ospedale principale al core switch del nuovo blocco operatorio; è in fibra il collegamento tra gli switch di core e gli switch di distribution; è in fibra il collegamento tra gli switch di distribution e gli switch di access. 

\paragraph{Cablaggio orizzontale}
Il cablaggio orizzontale collega gli swith di access al patch panel e collega i patch panel agli endpoint.  Questi collegamenti sono realizzati con i cavi di categoria 7 S/FTP con connettori GG45 (GigaGate 45, retrocompatibili con i classici connettori RJ45). I cavi cat7, con la loro velocità massima di 10Gigabit guardano al futuro e alla longevità dell'edificio, prevedendo un cablaggio strutturato valido per i prossimi venti anni ed oltre.
I collegamenti tra le torrette di endpoint e le postazioni di lavoro sono fatti con cavi di categoria 6.
\paragraph{} Nell'appendice \ref{appendix:planimetrie} sono mostrate le planimetrie del Nuovo Blocco Operatorio, che prenderemo come punto di riferimento per studiare la disposizione degli armadi di rete (MDF e IDF).
%\subsection{Armadi Rack}

\begin{figure}[h]
  \includegraphics[width=\linewidth]{Armadi.png}
  \caption{Organizzazione dell'armdio MDF e IDF.}
  \label{fig:armadi}
\end{figure}

\paragraph{Armadi} Come illustrato nel precedente capitolo \ref{subsec:nuovoblocco}, ogni armadio IDF è collegato all'armadio principale MDF. Nella figura \ref{fig:armadi} è presente un modello di organizzazione degli armadi Main ed Intermediate. L'armadio IDF è collegato tramite un doppio collegamento in fibra al core switch dell'armadio MDF. 


\paragraph{Distribuzione degli armadi sui quattro piani} Dall'analisi delle planimetrie del progetto del Nuovo Blocco Operatorio sono state individuate le aree migliori per la localizzazione dei reparti tecnici e l'ubicazione degli armadi che forniranno connettività alla parte del piano interessata. E' previsto un Main Distribution Frame che collega il nuovo blocco al core dell'ospedale ed è collegato ad ognuno dei sette  Intermediate Distribution Frame locati nei quattro piani. Sono state scelte posizioni interne alla struttura in locali senza finestre e condizionati. Nelle planimetrie di figura \ref{fig:p0r}, \ref{fig:p1r}, \ref{fig:p2r} e \ref{fig:p3r} sono evidenziate le posizioni degli armadi MDF e IDF. 
\begin{figure}[h]
  \includegraphics[width=\linewidth]{piano_0rack.png}
  \caption{Piano terreno con MDF e IDF.}
  \label{fig:p0r}
\end{figure}\begin{figure}[h]
  \includegraphics[width=\linewidth]{piano_1rack.png}
  \caption{Piano primo con IDF.}
  \label{fig:p1r}
\end{figure}
\begin{figure}[h]
  \includegraphics[width=\linewidth]{piano_2rack.png}
  \caption{Piano secondo con IDF.}
  \label{fig:p2r}
\end{figure}
\begin{figure}[h]
  \includegraphics[width=\linewidth]{piano_3rack.png}
  \caption{Piano terzo con IDF.}
  \label{fig:p3r}
\end{figure}
