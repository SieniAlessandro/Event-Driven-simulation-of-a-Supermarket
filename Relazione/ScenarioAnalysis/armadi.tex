\subsection{Armadi Rack}

\begin{figure}[h]
  \includegraphics[width=\linewidth]{Armadi.png}
  \caption{Organizzazione dell'armdio MDF e IDF.}
  \label{fig:armadi}
\end{figure}

\paragraph{Armadi} Come illustrato nel precedente capitolo \ref{subsec:nuovoblocco}, ogni armadio IDF è collegato all'armadio principale MDF. Nella figura \ref{fig:armadi} è presente un modello di organizzazione degli armadi Main ed Intermediate. L'armadio IDF è collegato tramite un doppio collegamento in fibra al core switch dell'armadio MDF. 


\paragraph{Distribuzione degli armadi sui quattro piani} Dall'analisi delle planimetrie del progetto del Nuovo Blocco Operatorio sono state individuate le aree migliori per la localizzazione dei reparti tecnici e l'ubicazione degli armadi che forniranno connettività alla parte del piano interessata. E' previsto un Main Distribution Frame che collega il nuovo blocco al core dell'ospedale ed è collegato ad ognuno dei sette  Intermediate Distribution Frame locati nei quattro piani. Sono state scelte posizioni interne alla struttura in locali senza finestre e condizionati. Nelle planimetrie di figura \ref{fig:p0r}, \ref{fig:p1r}, \ref{fig:p2r} e \ref{fig:p3r} sono evidenziate le posizioni degli armadi MDF e IDF. 
\begin{figure}[h]
  \includegraphics[width=\linewidth]{piano_0rack.png}
  \caption{Piano terreno con MDF e IDF.}
  \label{fig:p0r}
\end{figure}\begin{figure}[h]
  \includegraphics[width=\linewidth]{piano_1rack.png}
  \caption{Piano primo con IDF.}
  \label{fig:p1r}
\end{figure}
\begin{figure}[h]
  \includegraphics[width=\linewidth]{piano_2rack.png}
  \caption{Piano secondo con IDF.}
  \label{fig:p2r}
\end{figure}
\begin{figure}[h]
  \includegraphics[width=\linewidth]{piano_3rack.png}
  \caption{Piano terzo con IDF.}
  \label{fig:p3r}
\end{figure}
