\subsection{Virtual LAN}
\paragraph{} Una Virtual LAN è un metodo di creazione di reti logiche indipendenti all'interno della stessa rete fisica. Permette di segmentare il dominio di broadcast di una LAN, suddividendola in più reti LAN Virtuali. All'interno di uno switch si possono definire più VLAN e la stessa VLAN può essere dislocata su più switch. Utilizzare Virtual LAN consente di connettere host separati fisicamente alla stessa rete logica virtuale. Ciascuna VLAN si comporta come una LAN separata dalle altre, quindi, per la loro interconnessione si utilizza il routing di livello 3.
\paragraph{} I vantaggi dovuti all'utilizzo delle VALN sono molteplici:
\begin{itemize} 
	\item Aumento delle prestazioni e della sicurezza: aumentando il numero dei broadcast domain, riducendo la dimensione di ognuno si riduce il traffico di rete e si aumenta la sicurezza della stessa poichè gli host possono vedere solamente il traffico della loro VLAN e non quello delle altre. 
	\item Risparmio: realizzando le LAN Virtuali sulle stesse strutture fisiche si ottiene un notevole risparmio di tempo e di denaro, riducendo i requisiti hardware, essendo le reti separate logicamente e non fisicamente. 
	\item Flessibilità: le porte dello switch possono essere spostate da una VLAN ad un'altra per mezzo di semplici operazioni software, effettuabili da remoto; inoltre possono essere aggiunte altre VLAN utilizzando le porte esistenti.
\end{itemize}
\paragraph{} Il principale protocollo utilizzato per la configurazione delle VALN trunk è l'IEEE 802.1q, che descrive come possa essere partizionato il traffico su un singolo network fisico in più reti logiche, assegando un TAG ad ogni frame con dei byte aggiuntivi che identifichino la rete virtuale alla quale il pacchetto appartiene. 

\subsubsection{VLAN per il Nuovo Blocco Operatorio} 
\paragraph{} Per il Nuovo Blocco Operatorio si propone di suddividere la rete in 5 sottoreti differenti in base alla tipologia di utenza e al tipo di dispositivi.  Le VLAN realizzate sono 5:
\begin{itemize}
	\item \textbf{VLAN 10} ITManagement: per la sala server, il CED (Centro Elaborazione Dati) e per la gestione degli apparati di rete,
	\item \textbf{VLAN 20} Guest: rete ospiti affidata a ente esterno,
	\item \textbf{VLAN 100} VoIP: per i telefoni VoIP,
	\item \textbf{VLAN 150} SaleOperatorie: rete per l'accesso ad immagini cliniche ad alto livello di priorità,
	\item \textbf{VLAN 500} Fornitori: si prevede una VLAN per ogni fornitore, a partire dalla VLAN 501.
\end{itemize}
\paragraph{} Si è scelto di non utilizzare la VLAN 1 (di default le porte degli apparati Cisco appartengono
a questa) per motivi di sicurezza legati ad specifici attacchi informatici.
\subsubsection{Inter-VLAN routing and security}
\paragraph{} La rete ospiti non può accedere alle informazioni di nessuna delle altre reti, la rete delle sale operatorie deve poter accedere il più velocemente possibile alla rete dei server, così come il voip. Per poter soddisfare questi vincoli si effettuerà il ruoting inter-VLAN. Per risolvere il problema di avere tante interfacce fisiche quante sono le VLAN si utilizza la tecnica del "router on a stick". Questa tecnica utilizza le subinterfacce virtuali delle interfacce fisiche del router. Per il trasporto di più VLAN sullo stesso link, la porta dello switch collegata all’interfaccia fisica del router deve essere configurata in trunk 802.1Q. Perché tutto funzioni, anche le sub-interfaces devono essere configurate con l’encapsulation 802.1Q e deve essere configuato il routing tra le VLAN secondo le politiche sopra indicate. Lo schema della rete è illustrato in figura \ref{fig:vlan}.
\begin{figure}[ht]
  \includegraphics[width=\linewidth]{VLAN.png}
  \caption{Organizzazione della rete in Virtual LAN.}
  \label{fig:vlan}
\end{figure}


 








%\begin{table}[H]
%\centering
%\begin{tabular}{|l|l|l|}
%\hline
%\textbf{VLAN ID}        & \textbf{VLAN name}            & \textbf{Utilizzo}  \\ \hline
%1   					& Default    					& Default            \\ \hline
%2 						& Wifi              			& A follonica        \\ \hline
%3  						& Proxel           				& Bo??           	 \\ \hline
%10						& Dieci        	    			& Bo??           	 \\ \hline
%15 						& Replicazione      			& ??   			     \\ \hline
%20 					& Venti            				& Bo??               \\ \hline
%50  					& Failover		     			& copertura fallimenti  \\ \hline
%%100   				& Interna				        &                       \\ \hline
%200   				    & DMZ  					        & Demilitarized Zone    \\ \hline
%%250     				& DMZ WEB				        & Demilitarized Zone    \\ \hline
%300    				& RTRT 					        & Rete Telematica Regione Toscana      \\ \hline
%301     			    & INTER SST				        & Interna al Servizio Sanitario Toscana\\ \hline
%700   				    & test     					    &            \\ \hline
%1002    			    & fddi-default			        & automaticamente creata dallo switch\\ \hline
%1003  				    & token-ring-default	        & automaticamente creata dallo switch\\ \hline
%1004  				    & fddinet-default		        & automaticamente creata dallo switch\\ \hline
%1005     			    & trnet-default			        & automaticamente creata dallo switch\\ \hline

%\end{tabular}
%\caption{VLAN aziendali}
%\label{vlan}
%\end{table}


