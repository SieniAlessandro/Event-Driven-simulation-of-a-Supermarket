\subsection{DHCP}
\paragraph{} Ad oggi, nella rete Intranet dell'ex-ASL9 di Grosseto, gli indirizzi IP privati vengono assegnati staticamente, a cura del personale tecnico della UOS Infrastruttura Sud-Est, utilizzando una procedura web. 
\paragraph{} La scelta di assegnare indirizzo IP in modo statico:
\begin{itemize}
    \item determina un modesto onere gestionale, infatti l'assegnazione di indirizzi è necessaria solo in occasione dell'installazione in rete di nuove apparecchiature (PDL, stampanti multifunzione, apparati biomedicali);
    \item non costituisce un elemento di sicurezza per bloccare gli accessi indesiderati alla rete aziendale poichè non è difficile scoprire il piano di indirizzamneto di una sede, conoscere il default gateway e trovre un indirizzo IP non utilizzato tramite cui potersi connettere alla Intranet. 
\end{itemize} 
\paragraph{} L'implementazione di un servizio di assegnazione dinamica di indirizzi IP: 
\begin{itemize}
    \item eliminerebbe quasi completamente l'intervento umano in occasione dell'installazione in rete di nuovi apparati;
    \item abbasserebbe drasticamente il livello di sicurezza della rete poichè chiunque potrebbe connettersi ad un apparato di rete o ad una presa di rete connessa ad uno switch ed ottenere un indirizzo IP per accedere alla Intranet; inoltre chiunque connetta alla rete un apparato con servizio DHCP potrebbe distribuire indirizzi IP non coerenti con la rete aziendale, mandando fuori uso le apparecchiature.
 \end{itemize}
\paragraph{} Quest'ultimo problema è risolto combinando il servizio DHCP ad un metodo di autenticazione sicuro, come quello trattato nel capitolo \ref{sec:sicurezza} dell'IEEE 802.1x. E' ragionevole affermare che il servizio DHCP semplifica la configurazione degli apparati di rete, mentre l'implementazione di RADIUS e 802.1x costituisce la contromisura per l'apertura di falle nella sicurezza della rete, dovute proprio all'introduzione del servizio DHCP.

\subsubsection{Servizi offerti dal DHCP} 
Il protocollo DHCP prevede uno scambio di pacchetti tra un Client che fa richiesta di configurazione dinamica della rete ed un server tramite i pacchetti {\tt DHCP DISCOVEY, DHCP OFFER, DHCP REQUEST} e {\tt DHCP ACK}. Il server fornisce:
\begin{itemize}
	\item indirizzo IP prelevato dal pool di indirizzi disponibili,
	\item netmask della sottorete,
	\item indirizzo del Default Gateway,
	\item indirizzi dei Server DNS per la risoluzione di nomi di internet,
	\item dominio di default della rete aziendale,
	\item indirizzi dei Server WINS (Windows Internet Naming Service) per la risoluzione dei nomi NetBios,
	\item indirizzi dei Server NTP per la sincronizzazione dei clock tramite Network Time Protocol,
	\item indirizzi dei Server TFTP (Trivial File Transfer Protocol) per il download di immagini del sistema operativo (per telefoni VoIP),
	\item parametri di configurazione WPAD (Web Proxy Auto Discovery) del proxy server. 
\end{itemize}
Poter fornire dinamicamente la configurazione di rete agli apparati, permette di effettuare in modo relativamente semplice modifiche consistenti e complesse alla configurazione dei servizi di rete.

\subsubsection{Server DHCP} 
\paragraph{} Il client manda in modalità broadcast la richiesta iniziale di configurazione dinamica poichè non dispone di alcuna configurazione di rete per inviare un unicast ad uno specifico server. I pacchetti broadcast non vengono ovviamente inoltrati dai router. Per non dover installare un server DHCP in ogni rete locale remota e poter utilizzare un unico server DHCP centralizzato è necessario istruire i router di ogni singola sede affinchè reindirizzino le richiesta DHCP verso il server DHCP. 
\paragraph{} Il sistema operativo IOS dei router Cisco, che costituiscono l'infrastruttura di telecomunicazione RTRT3 prevede l'utilizzo del comando 
\begin{lstlisting}
#ip helper-address 
\end{lstlisting} 
seguito dall'indirizzo IP del server DHCP. In questo modo è possibile assegnare dinamicamente indirizzi IP in tutta la Intranet.
Il DHCP è stato testato nel laboratorio di Villa Pizzetti utilizzando un Server DHCP virtuale ed integrandolo all'implementazione del server RADIUS con 802.1x. Le versioni server utilizzate sono FreeRadius 2.2.6, MySQL 5.1.73 e DHCP 4.1.1.
Il laboratorio è illustrato in figura \ref{fig:lab}. 

\begin{figure}[ht]
  \includegraphics[width=\linewidth]{laboratorio_pizzetti.png}
  \caption{Schema del laboratorio di Villa Pizzetti.}
  \label{fig:lab}
\end{figure}

\subsubsection{Setup del server DHCP}  
La sottorete utilizzata a Villa Pizzetti è la 172.17.0.0/16 e tutti gli apparati della sede connessi alla rete hanno indirizzamento IP fisso.
Si è scelto prudenzialmente di configurare il server DHCP in modo che distribuisca indirizzi IP del pool 172.17.100.1 – 172.17.100.200. Gli indirizzi del pool non risultano assegnati ad alcun host, e se qualche host sulla rete Pizzetti richiedesse dinamicamente un indirizzo, otterrebbe un IP coerente con il piano di indirizzamento e potrebbe funzionare senza alcun problema. L'installazione e messa in esercizio del server DHCP è relativamente semplice. Dopo l'installazione, si edita il file di configurazione e si fa partire il servizio. \\


Questa è la configurazione utilizzata per i test nel file \verb|/etc/dhcp/dhcpd.conf| \\
\#PARAMETRI GLOBALI DEL SERVIZIO:
\begin{lstlisting}[numbers=left, breaklines=true]
#file nel quale il server scrive i lease asseganti
lease-file-name "/var/lib/dhcpd/dhcpd.leases";
#formato locale di data e ora da usare nel log
db-time-format local;
one-lease-per-client true;
ignore client-updates;
#43200 secondi = 12 ore
default-lease-time 43200;
#86400 secondi = 24 ore
max-lease-time 86400;
#questo parametro è definito per non fare interagire precauzionalmente il DHCP server con il DNS
ddns-update-style none;
#indica che il server è autoritativo e dovrebbe prevalere su altri eventuali server DHCP che operino sulla stessa rete
authoritative;

#il server scrive le proprie righe di log nel system log (/var/log/messages)
log-facility local0;
#il server esegue un ping verso l'indirizzo che sta per assegnare dinamicamente, se ottiene risposta, non lo assegna
ping-check true;
\end{lstlisting}
\\
\#OPZIONI GLOBALI INERENTI LA RETE AZIENDALE
\begin{lstlisting}[numbers=left, breaklines=true]
option domain-name "intra.asl9"; #Zona locale
option domain-name-servers 172.17.41.1, 172.16.94.97, 172.16.94.98; #Server DNS
option netbios-name-servers  172.17.41.1, 172.16.94.97, 172.16.94.98; #Sever WINS
option ntp-servers 172.16.94.5; #Server NTP
\end{lstlisting}
\\
\#FAILOVER
\begin{lstlisting}[numbers=left, breaklines=true]
failover peer "dhcp-failover" {
  primary;
  address 172.17.5.207;
  port 647;
  peer address 172.17.5.209;
  peer port 647;
  max-response-delay 30;
  max-unacked-updates 10;
  load balance max seconds 3;
  mclt 1800;
  split 256;
}
\end{lstlisting}
\\
\#PARAMETRI SPECIFICI PER I TEST DI VILLA PIZZETTI
\begin{lstlisting}[numbers=left, breaklines=true]
#definizione della subnet
subnet 172.17.0.0 netmask 255.255.0.0 {
        option subnet-mask 255.255.0.0;
        option broadcast-address 172.17.255.255;
        option routers 172.17.254.254;
        pool {
                failover peer "dhcp-failover";
                range 172.17.100.1 172.17.100.200;
        }
}
\end{lstlisting}
\\
\#Assegnazione dinamica dello stesso indirizzo IP a fronte dello stesso MAC address 
\begin{lstlisting}[numbers=left, breaklines=true]
host pi-acardia.intra.asl9 {
        hardware ethernet 00:1f:c6:87:42:b3;
        fixed-address 172.17.5.176;
}
\end{lstlisting}

\paragraph{} Quest'ultimo tipo di configurazione assicura che un host, settato per ottenere dinamicamnete la configurazione di rete, ottenga sempre lo stesso indirzzo IP a fornte dello stesso MAC address. E' un'emulazione di assegnazione di indirizzi IP fissi, effettuata però dal server DHCP. Ciò consente di settare su tutti gli apparati di rete di una sede la configurazione dinamica degli indirizzi e determinare quali apparati, in base al proprio MAC address, riceveranno dinamicamente sempre lo stesso indirizzo.
Questa tecnica è necessaria per gli apparati di rete che devono avere sempre lo stesso indirizzo come: 
\begin{itemize}
  \item stampanti multifunzione,
  \item terminali marcatempo,
  \item switch,
  \item access-point,
  \item ogni apparato che debba essere raggiunto da in sistema di monitoraggio,
  \item ogni apparato sul quale ci si debba loggare.
\end{itemize}

Quando verrà inserito nella rete un nuovo apparato che rientri nelle specificità precedenti si esegue questa proedura:
\begin{enumerate}
  \item si imposta la configurazoine della rete tramite DHCP
  \item si prende nota del MAC address
  \item si connette il dispositivo alla rete
  \item si consultà il file di lease del server DHCP per conoscere l'indirizzo IP assegnato al MAC address
  \item si modifica opportunamente il file di configurazione del server  DHCP in modo che l'apparato ottenga sempre lo stesso indirizzo IP in modo dinamico.
\end{enumerate}

